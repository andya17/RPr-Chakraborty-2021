% Options for packages loaded elsewhere
\PassOptionsToPackage{unicode}{hyperref}
\PassOptionsToPackage{hyphens}{url}
%
\documentclass[
]{article}
\usepackage{amsmath,amssymb}
\usepackage{iftex}
\ifPDFTeX
  \usepackage[T1]{fontenc}
  \usepackage[utf8]{inputenc}
  \usepackage{textcomp} % provide euro and other symbols
\else % if luatex or xetex
  \usepackage{unicode-math} % this also loads fontspec
  \defaultfontfeatures{Scale=MatchLowercase}
  \defaultfontfeatures[\rmfamily]{Ligatures=TeX,Scale=1}
\fi
\usepackage{lmodern}
\ifPDFTeX\else
  % xetex/luatex font selection
\fi
% Use upquote if available, for straight quotes in verbatim environments
\IfFileExists{upquote.sty}{\usepackage{upquote}}{}
\IfFileExists{microtype.sty}{% use microtype if available
  \usepackage[]{microtype}
  \UseMicrotypeSet[protrusion]{basicmath} % disable protrusion for tt fonts
}{}
\makeatletter
\@ifundefined{KOMAClassName}{% if non-KOMA class
  \IfFileExists{parskip.sty}{%
    \usepackage{parskip}
  }{% else
    \setlength{\parindent}{0pt}
    \setlength{\parskip}{6pt plus 2pt minus 1pt}}
}{% if KOMA class
  \KOMAoptions{parskip=half}}
\makeatother
\usepackage{xcolor}
\usepackage[margin=1in]{geometry}
\usepackage{color}
\usepackage{fancyvrb}
\newcommand{\VerbBar}{|}
\newcommand{\VERB}{\Verb[commandchars=\\\{\}]}
\DefineVerbatimEnvironment{Highlighting}{Verbatim}{commandchars=\\\{\}}
% Add ',fontsize=\small' for more characters per line
\usepackage{framed}
\definecolor{shadecolor}{RGB}{248,248,248}
\newenvironment{Shaded}{\begin{snugshade}}{\end{snugshade}}
\newcommand{\AlertTok}[1]{\textcolor[rgb]{0.94,0.16,0.16}{#1}}
\newcommand{\AnnotationTok}[1]{\textcolor[rgb]{0.56,0.35,0.01}{\textbf{\textit{#1}}}}
\newcommand{\AttributeTok}[1]{\textcolor[rgb]{0.13,0.29,0.53}{#1}}
\newcommand{\BaseNTok}[1]{\textcolor[rgb]{0.00,0.00,0.81}{#1}}
\newcommand{\BuiltInTok}[1]{#1}
\newcommand{\CharTok}[1]{\textcolor[rgb]{0.31,0.60,0.02}{#1}}
\newcommand{\CommentTok}[1]{\textcolor[rgb]{0.56,0.35,0.01}{\textit{#1}}}
\newcommand{\CommentVarTok}[1]{\textcolor[rgb]{0.56,0.35,0.01}{\textbf{\textit{#1}}}}
\newcommand{\ConstantTok}[1]{\textcolor[rgb]{0.56,0.35,0.01}{#1}}
\newcommand{\ControlFlowTok}[1]{\textcolor[rgb]{0.13,0.29,0.53}{\textbf{#1}}}
\newcommand{\DataTypeTok}[1]{\textcolor[rgb]{0.13,0.29,0.53}{#1}}
\newcommand{\DecValTok}[1]{\textcolor[rgb]{0.00,0.00,0.81}{#1}}
\newcommand{\DocumentationTok}[1]{\textcolor[rgb]{0.56,0.35,0.01}{\textbf{\textit{#1}}}}
\newcommand{\ErrorTok}[1]{\textcolor[rgb]{0.64,0.00,0.00}{\textbf{#1}}}
\newcommand{\ExtensionTok}[1]{#1}
\newcommand{\FloatTok}[1]{\textcolor[rgb]{0.00,0.00,0.81}{#1}}
\newcommand{\FunctionTok}[1]{\textcolor[rgb]{0.13,0.29,0.53}{\textbf{#1}}}
\newcommand{\ImportTok}[1]{#1}
\newcommand{\InformationTok}[1]{\textcolor[rgb]{0.56,0.35,0.01}{\textbf{\textit{#1}}}}
\newcommand{\KeywordTok}[1]{\textcolor[rgb]{0.13,0.29,0.53}{\textbf{#1}}}
\newcommand{\NormalTok}[1]{#1}
\newcommand{\OperatorTok}[1]{\textcolor[rgb]{0.81,0.36,0.00}{\textbf{#1}}}
\newcommand{\OtherTok}[1]{\textcolor[rgb]{0.56,0.35,0.01}{#1}}
\newcommand{\PreprocessorTok}[1]{\textcolor[rgb]{0.56,0.35,0.01}{\textit{#1}}}
\newcommand{\RegionMarkerTok}[1]{#1}
\newcommand{\SpecialCharTok}[1]{\textcolor[rgb]{0.81,0.36,0.00}{\textbf{#1}}}
\newcommand{\SpecialStringTok}[1]{\textcolor[rgb]{0.31,0.60,0.02}{#1}}
\newcommand{\StringTok}[1]{\textcolor[rgb]{0.31,0.60,0.02}{#1}}
\newcommand{\VariableTok}[1]{\textcolor[rgb]{0.00,0.00,0.00}{#1}}
\newcommand{\VerbatimStringTok}[1]{\textcolor[rgb]{0.31,0.60,0.02}{#1}}
\newcommand{\WarningTok}[1]{\textcolor[rgb]{0.56,0.35,0.01}{\textbf{\textit{#1}}}}
\usepackage{longtable,booktabs,array}
\usepackage{calc} % for calculating minipage widths
% Correct order of tables after \paragraph or \subparagraph
\usepackage{etoolbox}
\makeatletter
\patchcmd\longtable{\par}{\if@noskipsec\mbox{}\fi\par}{}{}
\makeatother
% Allow footnotes in longtable head/foot
\IfFileExists{footnotehyper.sty}{\usepackage{footnotehyper}}{\usepackage{footnote}}
\makesavenoteenv{longtable}
\usepackage{graphicx}
\makeatletter
\def\maxwidth{\ifdim\Gin@nat@width>\linewidth\linewidth\else\Gin@nat@width\fi}
\def\maxheight{\ifdim\Gin@nat@height>\textheight\textheight\else\Gin@nat@height\fi}
\makeatother
% Scale images if necessary, so that they will not overflow the page
% margins by default, and it is still possible to overwrite the defaults
% using explicit options in \includegraphics[width, height, ...]{}
\setkeys{Gin}{width=\maxwidth,height=\maxheight,keepaspectratio}
% Set default figure placement to htbp
\makeatletter
\def\fps@figure{htbp}
\makeatother
\setlength{\emergencystretch}{3em} % prevent overfull lines
\providecommand{\tightlist}{%
  \setlength{\itemsep}{0pt}\setlength{\parskip}{0pt}}
\setcounter{secnumdepth}{-\maxdimen} % remove section numbering
\ifLuaTeX
  \usepackage{selnolig}  % disable illegal ligatures
\fi
\IfFileExists{bookmark.sty}{\usepackage{bookmark}}{\usepackage{hyperref}}
\IfFileExists{xurl.sty}{\usepackage{xurl}}{} % add URL line breaks if available
\urlstyle{same}
\hypersetup{
  pdftitle={Learn GEE},
  pdfauthor={Joseph Holler},
  hidelinks,
  pdfcreator={LaTeX via pandoc}}

\title{Learn GEE}
\author{Joseph Holler}
\date{2023-09-28}

\begin{document}
\maketitle

This learning exercise is based upon Hanley et al (2003),
\url{https://doi.org/10.1093/aje/kwf215}

\hypertarget{learn-gee}{%
\subsection{Learn GEE}\label{learn-gee}}

\begin{itemize}
\tightlist
\item
  Avoid treating observations from the same ``cluster'' as independent
\item
  Examples of ``clusters'' include:

  \begin{itemize}
  \tightlist
  \item
    repeated measurements for the same person over time
  \item
    surveys of multiple members of the same household
  \item
    multiple observations in the same geographic region?
  \end{itemize}
\item
  Some approaches to resolving this problem include

  \begin{itemize}
  \tightlist
  \item
    discarding repeated observations
  \item
    including repeated observations but calculating statistical power
    based on the number of clusters rather than the number of
    observations
  \end{itemize}
\item
  How can you use all the data points while also not exaggerating your
  statistical power?
\item
  GEE uses ``weighted combinations of observations to extract the
  appropriate amount of information from correlated data''
\end{itemize}

Elements of notation:

\begin{longtable}[]{@{}ccc@{}}
\toprule\noalign{}
Variables & Parameter Symbol & Statistic Symbol \\
\midrule\noalign{}
\endhead
\bottomrule\noalign{}
\endlastfoot
household & \(h\) & \\
mean & \(\mu\) & \(\overline{y}\) \\
standard deviation & \(\delta\) & \\
proportion & \(P\) & \(p\) \\
regression coefficient & \(B\) & \(b\) \\
correlation coefficient & \(R\) & \(r\) \\
\end{longtable}

\hypertarget{variance-of-weighted-sum}{%
\subsection{Variance of Weighted Sum}\label{variance-of-weighted-sum}}

\begin{enumerate}
\def\labelenumi{\arabic{enumi}.}
\tightlist
\item
  Make a correlation matrix out of the variable and its weights
\item
  For each row/column combination, find: row-weight * column-weight *
  row-δ * column-δ * row-column-\(R\)
\item
  Sum the products from step 2
\end{enumerate}

\begin{itemize}
\tightlist
\item
  If there is no correlation, then off-diagonal R values are 0, and the
  weighted variance is simply the sum of squared weights and variances.
\item
  standard error of a mean is: \(\mu\) / \(\sqrt{n}\) \ldots{} and this
  is what you get if there is no correlation and equal weights in the
  sample
\item
  standard error of an estimate is basically the standard deviation of
  it\ldots{}
\item
  if data is correlated, the standard error increases (since you're no
  longer multiplying by 0 off-diagonal)
\end{itemize}

\hypertarget{what-if-some-observations-are-clustered}{%
\subsection{What if some observations are
clustered?}\label{what-if-some-observations-are-clustered}}

\begin{itemize}
\tightlist
\item
  three children: 1 single child and 2 siblings
\item
  optimal weight for observations correlated within clusters is: 1 / (1
  + \(r\)(\(k\)-1)) where \(k\) is the size of the cluster
\item
  evidence that this is the optimal weight is shown in Figure 3 and
  described as provable with calculus on pg 368
\item
  as correlation within clusters approaches 1, the effective sample size
  of the study decreases to the number of clusters. This make sense, as
  perfect correlation with clusters means that a cluster is effectively
  a single observation
\end{itemize}

\hypertarget{estimating-the-mean-and-correlation}{%
\subsection{Estimating the mean and
correlation}\label{estimating-the-mean-and-correlation}}

First, create sample data and calculate a matrix to use as a mask for
the off-diagonal products for calculating variance. Also calculate the
denominators for variance and covariance calculations.

\begin{Shaded}
\begin{Highlighting}[]
\CommentTok{\# create simple data }
\NormalTok{val }\OtherTok{\textless{}{-}} \FunctionTok{c}\NormalTok{(}\DecValTok{15}\NormalTok{, }\DecValTok{13}\NormalTok{, }\DecValTok{10}\NormalTok{, }\DecValTok{9}\NormalTok{, }\DecValTok{8}\NormalTok{)}
\NormalTok{grp }\OtherTok{\textless{}{-}} \FunctionTok{c}\NormalTok{(}\DecValTok{1}\NormalTok{, }\DecValTok{1}\NormalTok{, }\DecValTok{2}\NormalTok{, }\DecValTok{2}\NormalTok{, }\DecValTok{2}\NormalTok{)}
\NormalTok{fig4data }\OtherTok{\textless{}{-}} \FunctionTok{data.frame}\NormalTok{(val, grp)}
\FunctionTok{rm}\NormalTok{(val, grp)}

\NormalTok{n }\OtherTok{\textless{}{-}} \FunctionTok{length}\NormalTok{(fig4data}\SpecialCharTok{$}\NormalTok{val)}
\NormalTok{m }\OtherTok{\textless{}{-}} \FunctionTok{matrix}\NormalTok{(}\AttributeTok{nrow =}\NormalTok{ n, }\AttributeTok{ncol =}\NormalTok{ n)}

\NormalTok{fig4data }\OtherTok{\textless{}{-}}\NormalTok{ fig4data }\SpecialCharTok{\%\textgreater{}\%} 
  \FunctionTok{group\_by}\NormalTok{(grp) }\SpecialCharTok{\%\textgreater{}\%} 
  \FunctionTok{mutate}\NormalTok{(}\AttributeTok{k =} \FunctionTok{n}\NormalTok{(), }\AttributeTok{w =} \DecValTok{1} \SpecialCharTok{/}\NormalTok{ n) }\SpecialCharTok{\%\textgreater{}\%} 
  \FunctionTok{ungroup}\NormalTok{()}

\ControlFlowTok{for}\NormalTok{ (x }\ControlFlowTok{in} \DecValTok{1}\SpecialCharTok{:}\NormalTok{(n}\DecValTok{{-}1}\NormalTok{))\{}
  \ControlFlowTok{for}\NormalTok{(y }\ControlFlowTok{in}\NormalTok{ (x}\SpecialCharTok{+}\DecValTok{1}\NormalTok{)}\SpecialCharTok{:}\NormalTok{n)\{}
    \ControlFlowTok{if}\NormalTok{(fig4data[x,}\StringTok{"grp"}\NormalTok{] }\SpecialCharTok{==}\NormalTok{ fig4data[y,}\StringTok{"grp"}\NormalTok{])\{}
\NormalTok{      m[x,y] }\OtherTok{\textless{}{-}} \DecValTok{1}
\NormalTok{    \}}
\NormalTok{  \}}
\NormalTok{\}}

\NormalTok{var\_d }\OtherTok{\textless{}{-}}\NormalTok{ n }\SpecialCharTok{{-}} \DecValTok{1}
\NormalTok{covar\_d }\OtherTok{\textless{}{-}} \FunctionTok{sum}\NormalTok{(m, }\AttributeTok{na.rm =} \ConstantTok{TRUE}\NormalTok{) }\SpecialCharTok{{-}} \DecValTok{1}
\end{Highlighting}
\end{Shaded}

Here is a process for the estimation:

\begin{itemize}
\tightlist
\item
  start with assumption of no correlation \(R\) = 0
\item
  calculate estimate of \(\mu\) based on \(R\) = 0 (no auto-correlation)
\item
  calculate new estimate \(r\) of \(R\)
\item
  recalculate \(w\) weight for each observation as 1 / (1 +
  (\(k\)-1)\(r\))
\end{itemize}

\begin{Shaded}
\begin{Highlighting}[]
\NormalTok{reweight }\OtherTok{\textless{}{-}} \ControlFlowTok{function}\NormalTok{(k, r) \{}
    \CommentTok{\# new weight is 1 / (1 + (k{-}1)r)}
    \FunctionTok{return}\NormalTok{(}\DecValTok{1} \SpecialCharTok{/}\NormalTok{ (}\DecValTok{1} \SpecialCharTok{+}\NormalTok{ r }\SpecialCharTok{*}\NormalTok{ (k }\SpecialCharTok{{-}} \DecValTok{1}\NormalTok{)))}
\NormalTok{\}}
\end{Highlighting}
\end{Shaded}

\begin{itemize}
\tightlist
\item
  Repeat until you reach convergence (no more change in \(R\))
\end{itemize}

\begin{Shaded}
\begin{Highlighting}[]
\NormalTok{r\_prior }\OtherTok{\textless{}{-}} \DecValTok{1}
\NormalTok{r }\OtherTok{\textless{}{-}} \DecValTok{0}
\NormalTok{iteration }\OtherTok{\textless{}{-}} \DecValTok{1}

\ControlFlowTok{while}\NormalTok{(}\FunctionTok{abs}\NormalTok{(r }\SpecialCharTok{{-}}\NormalTok{ r\_prior) }\SpecialCharTok{\textgreater{}} \FloatTok{0.001}\NormalTok{)\{}
  \CommentTok{\# new weight is 1 / (1 + (k{-}1)r)}
\NormalTok{  fig4data }\OtherTok{\textless{}{-}}\NormalTok{ fig4data }\SpecialCharTok{\%\textgreater{}\%} \FunctionTok{mutate}\NormalTok{(}\AttributeTok{w =} \FunctionTok{reweight}\NormalTok{(fig4data}\SpecialCharTok{$}\NormalTok{k, r))}
  
  \CommentTok{\# weighted mean}
\NormalTok{  mu }\OtherTok{\textless{}{-}} \FunctionTok{sum}\NormalTok{(fig4data}\SpecialCharTok{$}\NormalTok{val }\SpecialCharTok{*}\NormalTok{ fig4data}\SpecialCharTok{$}\NormalTok{w) }\SpecialCharTok{/} \FunctionTok{sum}\NormalTok{(fig4data}\SpecialCharTok{$}\NormalTok{w)}
  
\NormalTok{  resid }\OtherTok{\textless{}{-}}\NormalTok{ fig4data}\SpecialCharTok{$}\NormalTok{val }\SpecialCharTok{{-}}\NormalTok{ mu}
  
  \CommentTok{\# estimated variance is sum of diagonal products divided by n(5) {-} 1}
\NormalTok{  v }\OtherTok{\textless{}{-}} \FunctionTok{sum}\NormalTok{(resid}\SpecialCharTok{\^{}}\DecValTok{2}\NormalTok{) }\SpecialCharTok{/}\NormalTok{ var\_d}
  
  \CommentTok{\# estimated covariance is sum of off{-}diagonal products divided by n(4) {-} 1}
\NormalTok{  c }\OtherTok{\textless{}{-}} \FunctionTok{sum}\NormalTok{(}\FunctionTok{outer}\NormalTok{(resid, resid, }\StringTok{"*"}\NormalTok{) }\SpecialCharTok{*}\NormalTok{ m, }\AttributeTok{na.rm =} \ConstantTok{TRUE}\NormalTok{) }\SpecialCharTok{/}\NormalTok{ covar\_d}
  
  \CommentTok{\# estimated correlation is estimated covariance / estimated variance}
\NormalTok{  r\_prior }\OtherTok{\textless{}{-}}\NormalTok{ r}
\NormalTok{  r }\OtherTok{\textless{}{-}}\NormalTok{ c }\SpecialCharTok{/}\NormalTok{ v}

  \FunctionTok{cat}\NormalTok{(}\StringTok{"iteration:"}\NormalTok{, iteration, }\StringTok{"}\SpecialCharTok{\textbackslash{}n}\StringTok{"}\NormalTok{,}
      \StringTok{"weighted mean:"}\NormalTok{, mu, }\StringTok{"}\SpecialCharTok{\textbackslash{}n}\StringTok{"}\NormalTok{,}
      \StringTok{"variance:"}\NormalTok{, v, }\StringTok{"}\SpecialCharTok{\textbackslash{}n}\StringTok{"}\NormalTok{,}
      \StringTok{"covariance:"}\NormalTok{, c, }\StringTok{"}\SpecialCharTok{\textbackslash{}n}\StringTok{"}\NormalTok{,}
      \StringTok{"correlation:"}\NormalTok{, r, }\StringTok{"}\SpecialCharTok{\textbackslash{}n}\StringTok{"}\NormalTok{,}
      \StringTok{"effective sample size:"}\NormalTok{, }\FunctionTok{sum}\NormalTok{(fig4data}\SpecialCharTok{$}\NormalTok{w), }\StringTok{"}\SpecialCharTok{\textbackslash{}n\textbackslash{}n}\StringTok{"}
\NormalTok{      )}
  
\NormalTok{  iteration }\OtherTok{\textless{}{-}}\NormalTok{ iteration }\SpecialCharTok{+} \DecValTok{1}
\NormalTok{\}}
\end{Highlighting}
\end{Shaded}

\begin{verbatim}
## iteration: 1 
##  weighted mean: 11 
##  variance: 8.5 
##  covariance: 6.333333 
##  correlation: 0.745098 
##  effective sample size: 5 
## 
## iteration: 2 
##  weighted mean: 11.43762 
##  variance: 8.739389 
##  covariance: 7.463922 
##  correlation: 0.8540553 
##  effective sample size: 2.350792 
## 
## iteration: 3 
##  weighted mean: 11.46677 
##  variance: 8.772337 
##  covariance: 7.557358 
##  correlation: 0.8614987 
##  effective sample size: 2.1865 
## 
## iteration: 4 
##  weighted mean: 11.46861 
##  variance: 8.774494 
##  covariance: 7.563345 
##  correlation: 0.8619694 
##  effective sample size: 2.17613
\end{verbatim}

\hypertarget{discussion-notes}{%
\subsubsection{Discussion notes}\label{discussion-notes}}

\begin{itemize}
\tightlist
\item
  For regression, you would use find the within-cluster correlation of
  regression residuals to re-define weights. Therefore, I wonder if the
  geepack algorithm can output the residuals or even the final set of
  weights?
\item
  Hierarchical \& multilevel models estimate between-cluster variation
  and incorporate this into standard errors, while GEE does the
  opposite---estimating within-cluster correlation.
\end{itemize}

Let's simulate the weights for observations within varying sizes of
clusters

\begin{Shaded}
\begin{Highlighting}[]
\CommentTok{\# create simple data }
\NormalTok{cluster\_size }\OtherTok{\textless{}{-}} \FunctionTok{c}\NormalTok{(}\DecValTok{1}\NormalTok{, }\DecValTok{2}\NormalTok{, }\DecValTok{3}\NormalTok{, }\FunctionTok{seq}\NormalTok{(}\DecValTok{5}\NormalTok{, }\DecValTok{20}\NormalTok{, }\AttributeTok{by =} \DecValTok{5}\NormalTok{), }\FunctionTok{seq}\NormalTok{(}\DecValTok{30}\NormalTok{, }\DecValTok{50}\NormalTok{, }\AttributeTok{by =} \DecValTok{10}\NormalTok{))}
\NormalTok{wbyk }\OtherTok{\textless{}{-}} \FunctionTok{data.frame}\NormalTok{(cluster\_size)}
\FunctionTok{rm}\NormalTok{(cluster\_size)}

\NormalTok{wbyk}\SpecialCharTok{$}\NormalTok{r0\_00 }\OtherTok{\textless{}{-}} \FunctionTok{round}\NormalTok{(}\FunctionTok{reweight}\NormalTok{(wbyk}\SpecialCharTok{$}\NormalTok{cluster\_size, }\DecValTok{0}\NormalTok{), }\DecValTok{2}\NormalTok{)}
\NormalTok{wbyk}\SpecialCharTok{$}\NormalTok{r0\_25 }\OtherTok{\textless{}{-}} \FunctionTok{round}\NormalTok{(}\FunctionTok{reweight}\NormalTok{(wbyk}\SpecialCharTok{$}\NormalTok{cluster\_size, }\FloatTok{0.25}\NormalTok{), }\DecValTok{2}\NormalTok{)}
\NormalTok{wbyk}\SpecialCharTok{$}\NormalTok{r0\_05 }\OtherTok{\textless{}{-}} \FunctionTok{round}\NormalTok{(}\FunctionTok{reweight}\NormalTok{(wbyk}\SpecialCharTok{$}\NormalTok{cluster\_size, }\FloatTok{0.5}\NormalTok{), }\DecValTok{2}\NormalTok{)}
\NormalTok{wbyk}\SpecialCharTok{$}\NormalTok{r1\_00 }\OtherTok{\textless{}{-}} \FunctionTok{round}\NormalTok{(}\FunctionTok{reweight}\NormalTok{(wbyk}\SpecialCharTok{$}\NormalTok{cluster\_size, }\DecValTok{1}\NormalTok{), }\DecValTok{2}\NormalTok{)}

\NormalTok{wbyk}
\end{Highlighting}
\end{Shaded}

\begin{verbatim}
##    cluster_size r0_00 r0_25 r0_05 r1_00
## 1             1     1  1.00  1.00  1.00
## 2             2     1  0.80  0.67  0.50
## 3             3     1  0.67  0.50  0.33
## 4             5     1  0.50  0.33  0.20
## 5            10     1  0.31  0.18  0.10
## 6            15     1  0.22  0.12  0.07
## 7            20     1  0.17  0.10  0.05
## 8            30     1  0.12  0.06  0.03
## 9            40     1  0.09  0.05  0.03
## 10           50     1  0.08  0.04  0.02
\end{verbatim}

\begin{itemize}
\tightlist
\item
  If there is no correlation within clusters, then the \(r\) will be 0
  or near 0 and the weights will be near 1
\item
  If there is perfect correlation, then \(r\) will approach 1 and the
  weights will be 1 / \texttt{cluster\ size}.
\item
  The implication is clear: if there is correlation within clusters,
  then then observations from small clusters weigh much more heavily
  than observations from large clusters. Keep in mind this is not
  correlation on a per cluster basis-- this is a single measure for
  within-cluster correlation for the whole dataset.
\item
  In this scenario, outliers in large clusters will have little effect
  on the model fitting.
\item
  However, outliers in small clusters will have an outsized effect on
  the model fit.
\end{itemize}

\end{document}
